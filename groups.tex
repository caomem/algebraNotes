 \documentclass[a4paper,12pt]{report}
\usepackage[a4paper,top=3cm,bottom=2cm,left=3cm,right=3cm,marginparwidth=1.75cm]{geometry}
\usepackage[brazil]{babel}
\usepackage[T1]{fontenc}
\usepackage[utf8]{inputenc}
\usepackage{amsmath}
\usepackage{amsthm}
\usepackage{MnSymbol}
\usepackage{wasysym}
\usepackage{hyperref}
\usepackage{color}
\definecolor{Blue}{rgb}{0,0,0.9}
\definecolor{Red}{rgb}{0.9,0,0}
\usepackage{esvect}
\usepackage{graphicx}
\usepackage{float}
\usepackage{indentfirst}
\usepackage{caption}
\usepackage{blkarray}
\newcommand\Mark[1]{\textsuperscript#1}
\usepackage{pgfplots}
\usepackage{amsfonts}
\usepackage[english, ruled, linesnumbered]{algorithm2e}
\usepackage{algorithmic}

\theoremstyle{plain}
\newtheorem{teorema}{Teorema}[section]
\newtheorem{lema}{Lema}[section]
\newtheorem{proposicao}{Proposição}[section]
\newtheorem{corolario}{Corolário}[section]

\theoremstyle{definition}
\newtheorem{definicao}{Definição}[section]
\newtheorem{observacao}{Observação}[section]
\newtheorem{exemplo}{Exemplo}[section]

\title{Teoria de Grupos: notas de estudo}
\author{Guilherme Philippi}
\begin{document}
\maketitle
\tableofcontents

\chapter{Grupos}
% Enunciar as principais referências desse capítulo

\section{Lei de composição}

\begin{definicao}[Lei de Composição]
	Uma \emph{Lei de Composição} sobre \(S\) é uma função \(F: S\times S \longrightarrow S\).
\end{definicao}


\begin{definicao}
	Para $a,b,c \; \in S$, uma Lei de Composição $F$ é dita
	
	\begin{itemize}
		\item \emph{Associativa} se \(F(F(a,b),c) = F(a,F(b,c))\);
		\item \emph{Comutativa} se \(F(a,b) = F(b,a)\).
	\end{itemize}
\end{definicao}

\begin{observacao}
	Usaremos a notação \(F(a,b) = ab\), para simplificar a escrita de
	propriedades.
\end{observacao}

\begin{proposicao}
	Seja uma lei associativa dada sobre o conjunto
	\(S\). Há uma única forma de definir, para todo inteiro \(n\), um
	produto de \(n\) elementos \(a_1,\dots,a_n \in S\) (diremos
	\([a_1\dotsb a_n]\)) com as seguintes propriedades:
	
	\begin{enumerate}
		\def\labelenumi{\arabic{enumi}.}
		\item
		o produto \([a_1]\) de um elemento é o próprio elemento;
		\item
		o produto \([a_1a_2]\) de dois elementos é dado pela lei de
		composição;
		\item
		para todo inteiro \(1\leq i\leq n\),
		\([a_1\dotsb a_n] = [a_1\dotsb a_i][a_{i+1}\dotsb a_n]\).
	\end{enumerate}
\end{proposicao}

\begin{proof}
	A demonstração dessa proposição é feita por indução em \(n\).
\end{proof}

\begin{definicao}
	Dizemos que \(e\in S\) é \emph{identidade} para uma lei de composição se \(ea = ae = a\) para todo \(a\in S\).
\end{definicao}

\begin{proposicao}
	O elemento identidade é único.
\end{proposicao}
\begin{proof}
	Se \(e,e'\) são identidades, já que \(e\) é identidade, então \(ee' = e'\) e, como $e'$ é uma identidade, \(ee' = e\). Logo \(e = e'\), isto é, a identidade é única.
\end{proof}

\begin{observacao}
	Usaremos $\vec{1}$ para representar a identidade multiplicativa e $\vec{0}$ para denotar a aditiva.
\end{observacao}

\begin{definicao}[Elemento inverso]
	Seja uma lei de composição que possua uma identidade. Um elemento \(a\in S\) é chamado \emph{invertível} se há um outro elemento \(b\in S\) tal que \(ab = ba = 1\). Desde que \(b\) exista, ela é única e a denotaremos por \(a^{-1}\) e a chamaremos
	\emph{inversa de $a$}.
\end{definicao}


\begin{proposicao}
	Se \(a,b\in S\) possuem inversa, então a composição \((ab)^{-1} = b^{-1}a^{-1}\).
\end{proposicao}

\begin{observacao}[Potências]
	Usaremos as seguintes notações:
	\begin{itemize}
		\item \(a^n = a^{n-1}a\) é a composição de \(a\dotsb a\) \(n\) vezes;
		\item \(a^{-n}\) é a inversa de \(a^n\);
		\item \(a^0 = \vec{1}\).
	\end{itemize}

	Com isso, tem-se que \(a^{r+s} = a^ra^s\) e \((a^r)^s = a^{rs}\). (Isso
	não induz uma notação de fração \(\frac{b}{a}\) a menos que seja uma lei
	comutativa, visto que \(ba^{-1}\) pode ser diferente de \(a^{-1}b\)).
	Para falar de uma lei de composição aditiva, usaremos \(-a\) no lugar de
	\(a^{-1}\) e \(na\) no lugar de \(a^n\).
\end{observacao}

\section{Grupos}

\begin{definicao}[Grupo]
	Um \emph{Grupo} é um conjunto \(G\) onde uma lei de
	composição associativa é dada sobre \(G\) tal que exista uma identidade
	e todo elemento possua uma inversa.
\end{definicao}

\begin{observacao}
	É comum abusar da notação e chamar um grupo e o conjunto de	seus elementos pelo mesmo simbolo, por exemplo, $G$.	
\end{observacao}

\begin{definicao}[Grupo abeliano]
	Um \emph{grupo abeliano} é um grupo com uma lei de
	composição comutativa. Costuma-se usar a notação aditiva para grupos
	abelianos.
\end{definicao}

\begin{proposicao}[Lei do cancelamento]
	Seja \(a,b,c\) elementos de um grupo \(G\). Se \(ab = ac\), então \(b = c\).
\end{proposicao} 

\section{Subgrupos}

\begin{definicao}[Subgrupo]
	Um subconjunto \(H\) de um grupo \(G\) é chamado de \emph{subgrupo} de \(G\) se possuir as seguintes propriedades:
	
	\begin{enumerate}
		\item \emph{(Fechado).} Se \(a,b\in H\), então \(ab\in H\);
		\item \emph{(Identidade).} \(1\in H\);
		\item \emph{(Inversível).} Se \(a\in H\), então \(a^{-1}\in H\).
	\end{enumerate}
	
\end{definicao}

\begin{observacao}[Lei de composição induzida]
	Veja que a propriedade 1. necessita de uma lei de composição. Usamos a
	lei de composição de \(G\) para definir uma lei de composição de \(H\),
	chamada \emph{lei de composição induzida}. Essas propriedades garantem
	que \(H\) é um grupo com respeito a sua lei induzida.
\end{observacao}

\begin{definicao}[Subgrupo apropriado]
	Todo grupo \(G\) possui dois subgrupos triviais: O subgrupo formado por
	todos os elementos de \(G\) e o subgrupo \(\{\vec{1}\}\), formado pela
	identidade de \(G\). Diz-se que um subgrupo é um \emph{subgrupo apropriado} se for diferente desses dois.
\end{definicao}

\begin{exemplo}
	Utilizando da notação multiplicativa, define-se o
	\emph{subgrupo cíclico \(H\)} gerados por um elemento arbitrário \(x\)
	de um grupo \(G\) como o conjunto de todas as potências de \(x\):
	\(H = \{\dots , x^{-2}, x^{-1},\vec{1},x,x^2,\dots\}\).
\end{exemplo}

\begin{definicao}
	Chama-se \emph{ordem} de um grupo \(G\) o número \(|G|\) de elementos de \(G\).
\end{definicao}

Também pode-se definir um subgrupo de um grupo \emph{\(G\) gerado por um
subconjunto \(U \subset G\)}. Esse é o menor subgrupo de \(G\) que
contém \(U\) e consiste de todos os elementos de \(G\) que podem ser
espressos como um produto de uma cadeia de elementos de \(U\) e seus
inversos.

\begin{exemplo}
	O \emph{grupo de quaternions \(H\)} é o menor subgrupo
	do conjunto de matrizes \(2\times 2\) complexas invertíveis que não é
	cíclico. Isso consiste nas oito matrizes
	\[H = \{\pm 1, \pm \mathbf{i}, \pm \mathbf{j}, \pm \mathbf{k}\},\] onde
	
	\[
	1=
	\begin{bmatrix}
		1 & 0 \\
		0 & 1 \\
	\end{bmatrix},
	\ \mathbf{i}=
	\begin{bmatrix}
		i & 0 \\
		0 & -i \\
	\end{bmatrix},
	\ \mathbf{j}=
	\begin{bmatrix}
		0 & 1 \\
		-1 & 0 \\
	\end{bmatrix},
	\ \mathbf{k}=
	\begin{bmatrix}
		0 & i \\
		i & 0 \\
	\end{bmatrix}.
	\]
	
	Os dois elementos \(\mathbf{i}, \mathbf{j}\) geram \(H\), e o calculo
	leva as formulas
	
	\[\mathbf{i}^4 = 1, \quad \mathbf{i}^2 = \mathbf{j}^2, \quad \mathbf{j}\mathbf{i} = \mathbf{i}^3\mathbf{j}.\]
\end{exemplo}


\phantomsection
\addcontentsline{toc}{section}{Referências}

\bibliographystyle{unsrt}
\bibliography{references}

\end{document}
