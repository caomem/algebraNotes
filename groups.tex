 \documentclass[a4paper,12pt]{report}
\usepackage[a4paper,top=3cm,bottom=2cm,left=3cm,right=3cm,marginparwidth=1.75cm]{geometry}
\usepackage[brazil]{babel}
\usepackage[T1]{fontenc}
\usepackage[utf8]{inputenc}
\usepackage{amsmath}
\usepackage{amsthm}
\usepackage{MnSymbol}
\usepackage{wasysym}
\usepackage{hyperref}
\usepackage{color}
\definecolor{Blue}{rgb}{0,0,0.9}
\definecolor{Red}{rgb}{0.9,0,0}
\usepackage{esvect}
\usepackage{graphicx}
\usepackage{float}
\usepackage{indentfirst}
\usepackage{caption}
\usepackage{blkarray}
\newcommand\Mark[1]{\textsuperscript#1}
\usepackage{pgfplots}
\usepackage{amsfonts}
\usepackage[english, ruled, linesnumbered]{algorithm2e}
\usepackage{algorithmic}

\newcommand{\nucleoe}{\emph{\text{nu }}}
\newcommand{\nucleo}{\text{nu }}
\newcommand{\imageme}{\emph{\text{nu }}}
\newcommand{\imagem}{\text{nu }}

\theoremstyle{plain}
\newtheorem{teorema}{Teorema}[section]
\newtheorem{lema}{Lema}[section]
\newtheorem{proposicao}{Proposição}[section]
\newtheorem{corolario}{Corolário}[section]

\theoremstyle{definition}
\newtheorem{definicao}{Definição}[section]
\newtheorem{observacao}{Observação}[section]
\newtheorem{exemplo}{Exemplo}[section]

\title{Teoria de Grupos: notas de estudo}
\author{Guilherme Philippi}
\begin{document}
\maketitle
\tableofcontents

\chapter{Grupos}
% Enunciar as principais referências desse capítulo

\section{Lei de composição}

\begin{definicao}[Lei de Composição]
	Uma \emph{Lei de Composição} sobre \(S\) é uma função \(F: S\times S \longrightarrow S\).
\end{definicao}


\begin{definicao}
	Para $a,b,c \; \in S$, uma Lei de Composição $F$ é dita
	
	\begin{itemize}
		\item \emph{Associativa} se \(F(F(a,b),c) = F(a,F(b,c))\);
		\item \emph{Comutativa} se \(F(a,b) = F(b,a)\).
	\end{itemize}
\end{definicao}

\begin{observacao}
	Usaremos a notação \(F(a,b) = ab\), para simplificar a escrita de
	propriedades.
\end{observacao}

\begin{proposicao}
	Seja uma lei associativa dada sobre o conjunto
	\(S\). Há uma única forma de definir, para todo inteiro \(n\), um
	produto de \(n\) elementos \(a_1,\dots,a_n \in S\) (diremos
	\([a_1\dotsb a_n]\)) com as seguintes propriedades:
	
	\begin{enumerate}
		\def\labelenumi{\arabic{enumi}.}
		\item
		o produto \([a_1]\) de um elemento é o próprio elemento;
		\item
		o produto \([a_1a_2]\) de dois elementos é dado pela lei de
		composição;
		\item
		para todo inteiro \(1\leq i\leq n\),
		\([a_1\dotsb a_n] = [a_1\dotsb a_i][a_{i+1}\dotsb a_n]\).
	\end{enumerate}
\end{proposicao}

\begin{proof}
	A demonstração dessa proposição é feita por indução em \(n\).
\end{proof}

\begin{definicao}
	Dizemos que \(e\in S\) é \emph{identidade} para uma lei de composição se \(ea = ae = a\) para todo \(a\in S\).
\end{definicao}

\begin{proposicao}
	O elemento identidade é único.
\end{proposicao}
\begin{proof}
	Se \(e,e'\) são identidades, já que \(e\) é identidade, então \(ee' = e'\) e, como $e'$ é uma identidade, \(ee' = e\). Logo \(e = e'\), isto é, a identidade é única.
\end{proof}

\begin{observacao}
	Usaremos $\vec{1}$ para representar a identidade multiplicativa e $\vec{0}$ para denotar a aditiva.
\end{observacao}

\begin{definicao}[Elemento inverso]
	Seja uma lei de composição que possua uma identidade. Um elemento \(a\in S\) é chamado \emph{invertível} se há um outro elemento \(b\in S\) tal que \(ab = ba = 1\). Desde que \(b\) exista, ela é única e a denotaremos por \(a^{-1}\) e a chamaremos
	\emph{inversa de $a$}.
\end{definicao}


\begin{proposicao}
	Se \(a,b\in S\) possuem inversa, então a composição \((ab)^{-1} = b^{-1}a^{-1}\).
\end{proposicao}

\begin{observacao}[Potências]
	Usaremos as seguintes notações:
	\begin{itemize}
		\item \(a^n = a^{n-1}a\) é a composição de \(a\dotsb a\) \(n\) vezes;
		\item \(a^{-n}\) é a inversa de \(a^n\);
		\item \(a^0 = \vec{1}\).
	\end{itemize}

	Com isso, tem-se que \(a^{r+s} = a^ra^s\) e \((a^r)^s = a^{rs}\). (Isso
	não induz uma notação de fração \(\frac{b}{a}\) a menos que seja uma lei
	comutativa, visto que \(ba^{-1}\) pode ser diferente de \(a^{-1}b\)).
	Para falar de uma lei de composição aditiva, usaremos \(-a\) no lugar de
	\(a^{-1}\) e \(na\) no lugar de \(a^n\).
\end{observacao}

\section{Grupos}

\begin{definicao}[Grupo]
	
	Um \emph{grupo} $(G,*)$ é um conjunto \(G\) onde uma lei de
	composição $*$ é dada sobre \(G\) tal que as seguintes propriedades são satisfeitas:
	
	\begin{enumerate}
		\item \emph{(Associatividade).} Para todo $a,b,c \in G$, tem-se $$(a*b)*c = a*(b*c);$$
		\item \emph{(Existência da identidade).} Existe um elemento $\vec{1}\in G$ tal que, para todo $a\in G$, $$\vec{1}*a = a*\vec{1} = a;$$
		\item \emph{(Existência do inverso).} Para todo $a\in G$ existe um elemento $a'\in G$ tal que $$a*a' = a'*a = \vec{1}.$$
	\end{enumerate}
\end{definicao}

\begin{observacao}
	É comum abusar da notação e chamar um grupo $(G,*)$ e o conjunto de	seus elementos $G$ pelo mesmo simbolo, omitindo a lei de composição quando não houver necessidade.	
\end{observacao}

\begin{definicao}[Grupo abeliano]
	Um \emph{grupo abeliano} é um grupo com uma lei de
	composição comutativa. Costuma-se usar a notação aditiva para grupos
	abelianos.
\end{definicao}

\begin{proposicao}[Lei do cancelamento]
	Seja \(a,b,c\) elementos de um grupo \(G\). Se \(ab = ac\), então \(b = c\).
\end{proposicao} 

\section{Subgrupos}

\begin{definicao}[Subgrupo]
	Um subconjunto \(H\) de um grupo \(G\) é chamado de \emph{subgrupo} de \(G\) (e escreve-se $H \leq G$) se possuir as seguintes propriedades:
	
	\begin{enumerate}
		\item \emph{(Fechado).} Se \(a,b\in H\), então \(ab\in H\);
		\item \emph{(Identidade).} \(1\in H\);
		\item \emph{(Inversível).} Se \(a\in H\), então \(a^{-1}\in H\).
	\end{enumerate}
	
\end{definicao}

\begin{observacao}[Lei de composição induzida]
	Veja que a propriedade 1 necessita de uma lei de composição. Usamos a
	lei de composição de \(G\) para definir uma lei de composição de \(H\),
	chamada \emph{lei de composição induzida}. Essas propriedades garantem
	que \(H\) é um grupo com respeito a sua lei induzida.
\end{observacao}

\begin{definicao}[Subgrupo apropriado]
	Todo grupo \(G\) possui dois subgrupos triviais: O subgrupo formado por
	todos os elementos de \(G\) e o subgrupo \(\{\vec{1}\}\), formado pela
	identidade de \(G\). Diz-se que um subgrupo é um \emph{subgrupo apropriado} se for diferente desses dois.
\end{definicao}

\begin{exemplo}
	Utilizando da notação multiplicativa, define-se o
	\emph{subgrupo cíclico \(H\)} gerados por um elemento arbitrário \(x\)
	de um grupo \(G\) como o conjunto de todas as potências de \(x\):
	\(H = \{\dots , x^{-2}, x^{-1},\vec{1},x,x^2,\dots\}\).
\end{exemplo}

\begin{definicao}
	Chama-se \emph{ordem} de um grupo \(G\) o número \(|G|\) de elementos de \(G\).
\end{definicao}

Também pode-se definir um subgrupo de um grupo \emph{\(G\) gerado por um
subconjunto \(U \subset G\)}. Esse é o menor subgrupo de \(G\) que
contém \(U\) e consiste de todos os elementos de \(G\) que podem ser
espressos como um produto de uma cadeia de elementos de \(U\) e seus
inversos.

\begin{exemplo}
	O \emph{grupo de quaternions \(H\)} é o menor subgrupo
	do conjunto de matrizes \(2\times 2\) complexas invertíveis que não é
	cíclico. Isso consiste nas oito matrizes
	\[H = \{\pm 1, \pm \mathbf{i}, \pm \mathbf{j}, \pm \mathbf{k}\},\] onde
	
	\[
	1=
	\begin{bmatrix}
		1 & 0 \\
		0 & 1 \\
	\end{bmatrix},
	\ \mathbf{i}=
	\begin{bmatrix}
		i & 0 \\
		0 & -i \\
	\end{bmatrix},
	\ \mathbf{j}=
	\begin{bmatrix}
		0 & 1 \\
		-1 & 0 \\
	\end{bmatrix},
	\ \mathbf{k}=
	\begin{bmatrix}
		0 & i \\
		i & 0 \\
	\end{bmatrix}.
	\]
	
	Os dois elementos \(\mathbf{i}, \mathbf{j}\) geram \(H\), e o calculo
	leva as formulas
	
	\[\mathbf{i}^4 = 1, \quad \mathbf{i}^2 = \mathbf{j}^2, \quad \mathbf{j}\mathbf{i} = \mathbf{i}^3\mathbf{j}.\]
\end{exemplo}

\section{Homomorfismos}

\begin{definicao}[Homomorfismo de grupo]
	Sejam \((G,*)\) e \((G',\cdot)\) dois grupos. Um \emph{homomorfismo} \(\varphi: G\longrightarrow G'\) é um mapeamento tal que
	\begin{equation}\tag{propriedade de homomorfismo}
		\varphi(a*b) = \varphi(a)\cdot\varphi(b), \; \forall \; a,b\; \in G.
	\end{equation}
	Quando isso acontece, dizemos que o mapeamento $\varphi$ \emph{preserva a estrutura algébrica de grupo}.

\end{definicao}

\begin{exemplo}[Inclusão]
	Seja \(H\) o subgrupo de um grupo \(G\). O homomorfismo \(i: H \longrightarrow G\) é dito \emph{inclusão} de \(H\) em \(G\), definido por \(i(x) = x\).
\end{exemplo}

\begin{proposicao}
	 Um homomorfismo \(\varphi: G\longrightarrow G'\) mapeia a identidade de $G$ à identidade de $G'$ e transforma as inversas de $G$ nas respectivas inversas em $G'$. Isto é, \(\varphi(\vec{1}) = \vec{1}\) e \(\varphi(a^{-1}) = \varphi(a)^{-1}\).
\end{proposicao}

\begin{definicao}[Imagem]
	A \emph{imagem} de um homomorfismo
	\(\varphi: G\longrightarrow G'\) é o subconjunto de \(G'\)	\[\text{im}\ \varphi = \{x\in G' \ |\ x = \varphi(a), \text{ para algum } a\in G\} = \varphi(G).\]
\end{definicao}

\begin{proposicao}
	A imagem de um homomorfismo $\varphi: G \longrightarrow G'$ é um subgrupo de $G'$.
\end{proposicao}

\begin{definicao}[Núcleo]
	O \emph{núcleo} do homomorfismo $\varphi: G \longrightarrow G'$ é o subconjunto de
	\(G\) formado pelos elementos que são mapeados pela identidade em
	\(G'\):	\[\text{nu} \ \varphi = \{a \in G \ | \ \varphi(a) = 1\} = \varphi^{-1}(1).\]
\end{definicao}

\begin{proposicao}
	O núcleo de um homomorfismo $\varphi: G \longrightarrow G'$ é um subgrupo de $G$.
\end{proposicao}

\section{Isomorfismos}

\begin{definicao}[Isomorfismo de grupos]
	Dois grupos \((G,*)\) e \((G',\cdot)\) são ditos \emph{isomórficos} se possuírem um homomorfismo bijetivo entre si, isto é, há um mapeamento \emph{bijetivo} $\varphi: G \longrightarrow G'$ (chamado \emph{relação de isomorfismo}) que respeita a propriedade de homomorfismo:
	\[\varphi(a*b) = \varphi(a)\cdot\varphi(b) \text{, para todo } a,b \in G.\] 
\end{definicao}

\begin{observacao}
	Usa-se a notação $G \approx G'$ para dizer que $G$ é isomorfo a $G'$.  
\end{observacao}

\begin{definicao}[Classe de isomorfismo]
	Diz-se que o conjunto de grupos isomórfos a um dado grupo \(G\) é a \emph{classe de isomorfismo de \(G\)}.	
\end{definicao}

\begin{proposicao}
	Qualquer dois grupos em uma mesma classe de isomorfismo também são isomorfos entre si.
\end{proposicao}

\begin{definicao}[Automorfismo]
	Quando uma relação de isomorfismo \(\varphi: G\longrightarrow G\) é definida de um grupo \(G\) para ele mesmo,	chamamos esse tipo de isomorfismo de \emph{automorfismo} de \(G\).
\end{definicao}

\begin{exemplo}[Conjugação]
	 Seja \(b\in G\) um elemento fixo. Então, a
	\emph{conjugação de \(G\) por \(b\)} é o mapeamento \(\varphi\) de \(G\)
	para ele mesmo definido por
	\[\varphi_b(x) = bxb^{-1}.\]
	Esse é um automorfismo porque:
	\begin{itemize}
		\item é compatível com a multiplicação no
		grupo: \[\varphi_b(xy) = bxyb^{-1} = bx\vec{1}yb^{-1} = bxb^{-1}byb^{-1} = \varphi_b(x)\varphi_b(y);\]
		\item é um mapa bijetivo visto que existe a função inversa $\varphi_b^{-1}(x) = b^{-1}xb = \varphi_{b^{-1}}(x)$ (isto é, a conjugação por \(b^{-1}\)) que, de forma análoga, também é compatível com a multiplicação no grupo.
	\end{itemize}
\end{exemplo}

\begin{observacao}
	Se o grupo é abeliano, a conjugação é o mapa identidade:
	\(bab^{ -1} = abb^{-1} = a\). Porém, qualquer grupo não comutativo tem
	alguma conjugação não trivial, logo possui também um automorfismo não
	trivial. 
\end{observacao}

\begin{definicao}[Conjugado]
	O elemento \(bab^{-1}\) é chamado \emph{conjugado de \(a\) por \(b\)}. Dois elementos \(a, a'\in G\) são ditos \emph{conjugados} se existe \(b\in G\) tal que \(a' = bab^{-1}\).	
\end{definicao}

\begin{observacao}
	O conjugado tem uma interpretação muito útil: Se escrevermos
	\(bab^{-1}\) como \(a'\), então \[ba = a'b.\] Ou seja, pode-se pensar na
	conjugação como a mudança em \(a\) que resulta de mover \(b\) de um lado
	para o outro na equação.
\end{observacao}		


\begin{proposicao}
	Se \(a\in \nucleoe\varphi\) e \(b\) é qualquer elemento do grupo \(G\), então o conjugado \(bab^{-1} \in \nucleoe\varphi\).
\end{proposicao}

\begin{definicao}[Subgrupo normal]
	Um subgrupo \(N\) de um grupo \(G\) é chamado \emph{subgrupo normal} se para cada \(a\in N\) e \(b\in G\), o conjugado
	\(bab^{-1} \in N\).
\end{definicao}


Fica claro que o núcleo de um homomorfismo é um subgrupo normal. Além
disso, todo subgrupo de um grupo abeliano também é um subgrupo normal
(se \(G\) é abeliano, então \(bab^{-1} = a\)). Mas isso não é
necessáriamente verdade em subgrupos de grupos não abelianos.

\begin{definicao}[Centro de um grupo]
	O \emph{centro} \(Z(G)\) de um grupo \(G\) é o
	conjunto de elementos que comutam com todo elemento de \(G\):
	\[Z(G) = \{z \in G \ | \ zx = xz \text{ para todo } x \in G\}.\]
\end{definicao}

\begin{proposicao}
	O centro de todo grupo é um subgrupo normal do grupo.
\end{proposicao}


\phantomsection
\addcontentsline{toc}{chapter}{Referências Bibliográficas}

\bibliographystyle{unsrt}
\bibliography{references}

\end{document}
