\documentclass[a4paper,12pt]{article}
\usepackage[a4paper,top=3cm,bottom=2cm,left=3cm,right=3cm,marginparwidth=1.75cm]{geometry}
\usepackage[brazil]{babel}
\usepackage[T1]{fontenc}
\usepackage[utf8]{inputenc}
\usepackage{amsmath}
\usepackage{amsthm}
\usepackage{MnSymbol}
\usepackage{wasysym}
\usepackage{hyperref}
\usepackage{color}
\definecolor{Blue}{rgb}{0,0,0.9}
\definecolor{Red}{rgb}{0.9,0,0}
\usepackage{esvect}
\usepackage{graphicx}
\usepackage{float}
\usepackage{indentfirst}
\usepackage{caption}
\usepackage{blkarray}
\newcommand\Mark[1]{\textsuperscript#1}
\usepackage{pgfplots}
\usepackage{amsfonts}
\usepackage[english, ruled, linesnumbered]{algorithm2e}
\usepackage{algorithmic}
\usepackage{enumitem}

\newcommand{\nucleoe}{\emph{\text{nu }}}
\newcommand{\nucleo}{\text{nu }}
\newcommand{\imageme}{\emph{\text{nu }}}
\newcommand{\imagem}{\text{nu }}

\theoremstyle{plain}
\newtheorem{teorema}{Teorema}[section]
\newtheorem{lema}{Lema}[section]
\newtheorem{proposicao}{Proposição}[section]
\newtheorem{corolario}{Corolário}[section]

\theoremstyle{definition}
\newtheorem{definicao}{Definição}[section]
\newtheorem{observacao}{Observação}[section]
\newtheorem{exemplo}{Exemplo}[section]

\newenvironment{solucao}
{\renewcommand\qedsymbol{$\triangle$}\begin{proof}[Solução]}{\end{proof}}

\title{\textsc{Homomorfismos de Anéis}\\ \textsl{um resumo}}
\author{Guilherme Philippi}
\begin{document}
	\maketitle
	
	Esse texto pretende ser uma introdução aos conceitos fundamentais entorno de homomorfismos de anéis. Tudo que aqui se apresenta fora extraído de \cite{johnAlgebra, michalAlgebra, dominguesAlgebra}, principalmente de \cite{dominguesAlgebra}.
	
	\section{Grupos}
	
	\begin{definicao}[Grupo]
		Um \emph{grupo} $(G,*)$ é um conjunto \(G\) onde uma lei de
		composição $*$ é dada sobre \(G\) tal que os seguintes axiomas são satisfeitos:
		
		\begin{enumerate}
			\item \emph{(Associatividade).} Para todo $a,b,c \in G$, tem-se $$(a*b)*c = a*(b*c);$$
			\item \emph{(Existência da identidade).} Existe um elemento $\vec{1}\in G$ tal que, para todo $a\in G$, $$\vec{1}*a = a*\vec{1} = a;$$
			\item \emph{(Existência do inverso).} Para todo $a\in G$ existe um elemento $a'\in G$ tal que $$a*a' = a'*a = \vec{1}.$$
		\end{enumerate}
	\end{definicao}

	\begin{observacao}[Notação]
		É comum abusar da notação e chamar um grupo $(G,*)$ e o conjunto de	seus elementos $G$ pelo mesmo simbolo, omitindo a lei de composição, na falta de ambiguidade. Também, quando não houver ambiguidade, suprimiremos o simbolo da lei, fazendo $a*b = ab$.
	\end{observacao}
	
	\begin{definicao}[Grupo abeliano]
		Um \emph{grupo abeliano} é um grupo $G$ com uma \emph{lei de
		composição comutativa}, isto é, $ab = ba$, para todo $a,b \in G$.
	\end{definicao}

	\begin{proposicao}[Lei do cancelamento]
		Seja \(a,b,c\) elementos de um grupo \(G\). Se \(ab = ac\), então \(b = c\).
	\end{proposicao} 

	\section{Anéis}
	
	\begin{definicao}[Anel]
		Um \emph{anel} $(R, +, \cdot)$ é um conjunto $R$ acompanhado de duas operações binárias $+$ e $\cdot$ definidas sobre $R$ tais que os seguintes axiomas são satisfeitos:
		\begin{enumerate}
			\item $(R, +)$ é um grupo abeliano.
			\item A operação $\cdot$ é associativa.
			\item Para todo $a,b,c\in R$ vale a \emph{lei da distributividade à esquerda} e a \emph{lei de distributividade à direita}, respectivamente, $$a\cdot(b+c) = (a\cdot b) + (a\cdot c)\quad \text{e} \quad (a+b)\cdot c = (a\cdot c) + (b\cdot c).$$
		\end{enumerate}
	\end{definicao}
	
	\begin{exemplo}
		Todo subconjunto dos números complexos que é fechado para a adição e multiplicação usual dos complexos é um anel. Por exemplo, $(\mathbb{Z}, +, \cdot)$, $(\mathbb{Q}, +, \cdot)$, $(\mathbb{R}, +, \cdot)$ e $(\mathbb{C}, +, \cdot)$ são todos anéis.
	\end{exemplo}
	
	\begin{observacao}[Notação]
		 Da mesma forma que com os grupos, costuma-se denotar o anel $(R, +, \cdot)$ apenas por seu conjunto $R$. Também, para um anel $(R, +, \cdot)$, chama-se sua primeira operação $+$ de \emph{adição do anel} e sua segunda operação $\cdot$ de \emph{multiplicação do anel}. 
	\end{observacao}

	\begin{proposicao}
		Se $R$ é um anel com identidade aditiva $\vec 0$, então, $\forall a \in R$, $$\vec 0 \cdot a = a \cdot \vec 0 = \vec 0.$$
	\end{proposicao}
	\begin{proof}
		Como $(R, +)$ é um grupo abeliano, tem-se que $$a\vec 0 + a\vec 0 = a(\vec 0 + \vec 0) = a\vec 0 = \vec 0 + a\vec 0.$$ E, pela lei de cancelamento do grupo, $$a\vec0 + a\vec 0 = \vec 0 + a\vec 0 \ \implies \ a\vec 0 = \vec 0.$$
		De forma semelhante, $$\vec 0a + \vec 0a = (\vec 0 + \vec 0)a = \vec 0a = \vec 0 + \vec 0a \ \implies \ \vec 0a = \vec 0.$$ Daí, segue que $a\vec 0 = \vec 0a = \vec 0$. 
	\end{proof}

	\begin{proposicao}
		Se $R$ é um anel, então, para todo $a,b\in R$ vale
		\begin{enumerate}
			\item $a(-b) = (-a)b = -(ab)$ e
			\item $(-a)(-b) = ab.$
		\end{enumerate}
	\end{proposicao}

	\section{Homomorfismos de anéis}
	
	\begin{definicao}[Homomorfismo de anéis]
		Sejam dois anéis $(R, +, \cdot)$ e $(R', +', \cdot')$. Um mapa $\phi: R \longrightarrow R'$ é um \emph{homomorfismo} se a \emph{propriedade de homomorfismo} vale para ambas as operações, isso é, se, para todo $a,b\in R$,
				$$\phi(a+b) = \phi(a) +' \phi(b) \quad \text{e} \quad \phi(a\cdot b) = \phi(a) \cdot' \phi(b).$$
	\end{definicao}
	
	\begin{exemplo}[Homomorfismo trivial]
		Sejam os anéis $R$, $R'$ e o elemento neutro $\vec 0$ da adição do anel $R'$. A aplicação $\phi: R\longrightarrow R'$ definida por $\phi(a) = \vec 0$, para todo $a\in R$, é um homomorfismo de anéis porque $$\phi(a + b) = \vec 0 = \vec 0 \ +' \vec 0 = f(a) +' f(b) \quad \text{e} \quad f(a\cdot b) = \vec 0 = \vec 0 \ \cdot' \vec 0 = f(a)\ \cdot' f(b).$$
		A essa aplicação dá-se o nome \emph{homomorfismo trivial de anéis}.
	\end{exemplo}
	
	\begin{definicao}[Homomorfismo injetivo e sobrejetivo]
		Chama-se de \emph{homomorfismo injetivo} e \emph{homomorfismo sobrejetivo} um homomorfismo de anéis definido, respectivamente, por uma função injetiva ou uma função sobrejetiva.
	\end{definicao}

	\begin{exemplo}
		Seja o homomorfismo de anéis $\phi: \mathbb{Z} \longrightarrow \mathbb{Z}\times \mathbb{Z}$ tal que $\phi(n) = (n,0)$, para todo $n\in\mathbb{Z}$. Perceba que, para cada $(n,0)\in \mathbb{Z}\times \mathbb{Z}$ tem-se um único $n\in \mathbb{Z}$ tal que $\phi(n) = (n,0)$, daí, $\phi$ é injetiva e esse é um homomorfismo injetivo. Também, seja $\mu:\mathbb{Z}\times\mathbb{Z} \longrightarrow \mathbb{Z}$ o homomorfismo tal que $\mu(n,m) = n$ para todo $(n,m)\in \mathbb{Z}\times \mathbb{Z}$. É fácil perceber que para todo $z\in \mathbb{Z}$, existirá $(z,0)\in \mathbb{Z}\times \mathbb{Z}$, donde $\mu$ é um homomorfismo sobrejetivo.
	\end{exemplo}
	
	\begin{proposicao}
		Se $\phi: R \longrightarrow R'$ é um homomorfismo de anéis, então, para todo $a,b\in A$,
		\begin{itemize}
			\item $\phi(0_R) = 0_{R'}$,
			\item $\phi(-a) = -\phi(a)$ e
			\item $\phi(a-b) = \phi(a) - \phi(b)$.
		\end{itemize} 	
	\end{proposicao}
	\begin{proof}
		Como $\phi(a) = \phi(a+0_R) = \phi(a)+\phi(0_R)$, pela propriedade de homomorfismo, então, $$\phi(a) = \phi(a)+\phi(0_R) \implies -\phi(a)+\phi(a) = -\phi(a)+\phi(a)+\phi(0_R),$$ isto é, $0_{R'} = \phi(0_R)$.
	\\
	
		\noindent Daí segue que, $$0_{R'} = \phi(0_R) = \phi(a - a) = \phi(a) + \phi(-a),$$
		e como $0_{R'} = \phi(a) + \phi(-a)$, $$\phi(-a) = -\phi(a).$$
		\noindent Fica evidente que $$\phi(a-b) = \phi(a) + \phi(-b) = \phi(a) - \phi(b).$$
	\end{proof}
	
	\begin{proposicao}
		Seja $\phi:R \longleftarrow	R'$ um homomorfismo de anéis onde $1_R \in R$ é identidade do produto de $R$. Então
		\begin{itemize}
			\item $R'$ possui identidade multiplicativa $1_{R'}$ e $\phi(1_R) = 1_{R'}$;
			\item se $a\in R$ possui inversa multiplicativa $a^{-1}$, então $\phi(a)^{-1} = \phi(a^{-1})$.
		\end{itemize}	
	\end{proposicao}

	\begin{definicao}[subanel]
		
	\end{definicao}
	
	
	\begin{proposicao}
		Se $\phi: R \longrightarrow R'$ é um homomorfismo de anéis e $S\leq R$
	\end{proposicao}
	
	
	\phantomsection
	\addcontentsline{toc}{chapter}{Referências Bibliográficas}
	
	\bibliographystyle{unsrt}
	\bibliography{../../references}
	
\end{document}