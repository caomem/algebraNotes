\documentclass[a4paper,12pt]{article}
\usepackage[a4paper,top=3cm,bottom=2cm,left=3cm,right=3cm,marginparwidth=1.75cm]{geometry}
\usepackage[brazil]{babel}
\usepackage[T1]{fontenc}
\usepackage[utf8]{inputenc}
\usepackage{amsmath}
\usepackage{amsthm}
\usepackage{MnSymbol}
\usepackage{wasysym}
\usepackage{hyperref}
\usepackage{color}
\definecolor{Blue}{rgb}{0,0,0.9}
\definecolor{Red}{rgb}{0.9,0,0}
\usepackage{esvect}
\usepackage{graphicx}
\usepackage{float}
\usepackage{indentfirst}
\usepackage{caption}
\usepackage{blkarray}
\newcommand\Mark[1]{\textsuperscript#1}
\usepackage{pgfplots}
\usepackage{amsfonts}
\usepackage[english, ruled, linesnumbered]{algorithm2e}
\usepackage{algorithmic}
\usepackage{enumitem}

\newcommand{\nucleoe}{\emph{\text{nu }}}
\newcommand{\nucleo}{\text{nu }}
\newcommand{\imageme}{\emph{\text{nu }}}
\newcommand{\imagem}{\text{nu }}

\theoremstyle{plain}
\newtheorem{teorema}{Teorema}[section]
\newtheorem{lema}{Lema}[section]
\newtheorem{proposicao}{Proposição}[section]
\newtheorem{corolario}{Corolário}[section]

\theoremstyle{definition}
\newtheorem{definicao}{Definição}[section]
\newtheorem{observacao}{Observação}[section]
\newtheorem{exemplo}{Exemplo}[section]

\newenvironment{solucao}
{\renewcommand\qedsymbol{$\triangle$}\begin{proof}[Solução]}{\end{proof}}

\title{\textsc{Homomorfismos de Anéis}\\ \textsl{um resumo}}
\author{Guilherme Philippi}
\begin{document}
	\maketitle
	
	Esse texto pretende ser uma introdução aos conceitos fundamentais entorno de homomorfismos de anéis. Tudo que aqui se apresenta fora extraído de \cite{johnAlgebra, michalAlgebra, dominguesAlgebra}, principalmente de \cite{dominguesAlgebra}.
	
	\section{Anéis}
	
	\begin{definicao}[Grupo]
		Um \emph{grupo} $(G,*)$ é um conjunto \(G\) onde uma lei de
		composição $*$ é dada sobre \(G\) tal que os seguintes axiomas são satisfeitos:
		
		\begin{enumerate}
			\item \emph{(Associatividade).} Para todo $a,b,c \in G$, tem-se $$(a*b)*c = a*(b*c);$$
			\item \emph{(Existência da identidade).} Existe um elemento $\vec{1}\in G$ tal que, para todo $a\in G$, $$\vec{1}*a = a*\vec{1} = a;$$
			\item \emph{(Existência do inverso).} Para todo $a\in G$ existe um elemento $a'\in G$ tal que $$a*a' = a'*a = \vec{1}.$$
		\end{enumerate}
	\end{definicao}
	
	\begin{definicao}[Grupo abeliano]
		Um \emph{grupo abeliano} é um grupo $G$ com uma \emph{lei de
		composição comutativa}, isto é, $a*b = b*a$, para todo $a,b \in G$.
	\end{definicao}
	
	\begin{definicao}[Anel]
		Um \emph{anel} $(R, +, \cdot)$ é um conjunto $R$ acompanhado de duas operações binárias $+$ e $\cdot$ definidas sobre $R$ tais que os seguintes axiomas são satisfeitos:
		\begin{enumerate}
			\item $(R, +)$ é um grupo abeliano.
			\item A operação $\cdot$ é associativa.
			\item Para todo $a,b,c\in R$, valem a \emph{lei da distributividade à esquerda} $$a\cdot(b+c) = (a\cdot b) + (a\cdot c)$$ e a \emph{lei de distributividade à direita} $$(a+b)\cdot c = (a\cdot c) + (b\cdot c).$$
		\end{enumerate}
	\end{definicao}
	
	\begin{observacao}[Notação]
		É comum abusar da notação e chamar um grupo $(G,*)$ e o conjunto de	seus elementos $G$ pelo mesmo simbolo, omitindo a lei de composição na falta de ambiguidade. Da mesma forma, costuma-se denotar o anel $(R, +, \cdot)$ apenas por seu conjunto $R$.
	\end{observacao}
	
	\section{Homomorfismos de anéis}
	
	\begin{definicao}[Homomorfismo de anéis]
		Sejam dois anéis $(R, +, \cdot)$ e $(R', +', \cdot')$. Um mapa $\phi: R \longrightarrow R'$ é um \emph{homomorfismo} se a \emph{propriedade de homomorfismo} vale para ambas as operações, isso é, se, para todo $a,b\in R$,
				$$\phi(a+b) = \phi(a) +' \phi(b) \text{ e } \phi(a\cdot b) = \phi(a) \cdot' \phi(b).$$
	\end{definicao}
	
	
	\phantomsection
	\addcontentsline{toc}{chapter}{Referências Bibliográficas}
	
	\bibliographystyle{unsrt}
	\bibliography{../../references}
	
\end{document}