\documentclass[a4paper,12pt]{article}
\usepackage[a4paper,top=3cm,bottom=2cm,left=3cm,right=3cm,marginparwidth=1.75cm]{geometry}
\usepackage[brazil]{babel}
\usepackage[T1]{fontenc}
\usepackage[utf8]{inputenc}
\usepackage{amsmath}
\usepackage{amsthm}
\usepackage{MnSymbol}
\usepackage{wasysym}
\usepackage{hyperref}
\usepackage{color}
\definecolor{Blue}{rgb}{0,0,0.9}
\definecolor{Red}{rgb}{0.9,0,0}
\usepackage{esvect}
\usepackage{graphicx}
\usepackage{float}
\usepackage{indentfirst}
\usepackage{caption}
\usepackage{blkarray}
\newcommand\Mark[1]{\textsuperscript#1}
\usepackage{pgfplots}
\usepackage{amsfonts}
\usepackage[english, ruled, linesnumbered]{algorithm2e}
\usepackage{algorithmic}
\usepackage{enumitem}

\newcommand{\nucleoe}{\emph{\text{nu }}}
\newcommand{\nucleo}{\text{nu }}
\newcommand{\imageme}{\emph{\text{nu }}}
\newcommand{\imagem}{\text{nu }}

\theoremstyle{plain}
\newtheorem{teorema}{Teorema}[section]
\newtheorem{lema}{Lema}[section]
\newtheorem{proposicao}{Proposição}[section]
\newtheorem{corolario}{Corolário}[section]

\theoremstyle{definition}
\newtheorem{definicao}{Definição}[section]
\newtheorem{observacao}{Observação}[section]
\newtheorem{exemplo}{Exemplo}[section]

\newenvironment{solucao}
{\renewcommand\qedsymbol{$\triangle$}\begin{proof}[Solução]}{\end{proof}}

\title{\textsc{Operações Binárias e Aplicaćões}\\ \textsl{um resumo}}
\author{Guilherme Philippi}
\begin{document}
	\maketitle
	
	Apresenta-se nesse texto um compilado de definições e resultados envolvendo os conceitos de operações binárias e algumas de suas aplica. Tudo que aqui se apresenta fora extraído de \cite{johnAlgebra, michalAlgebra, maierAlgebra}, principalmente de \cite{johnAlgebra}.

\section{Operações binárias}

\begin{definicao}[Operação binária]
	Uma \emph{operação binária} sobre um conjunto \(S\) é uma função \(*: S\times S \longrightarrow S\).
\end{definicao}

\begin{exemplo}[Produto sobre $\mathbb{R}$]
	Seja a função $\cdot: \mathbb{R}\times\mathbb{R}\longrightarrow\mathbb{R}$ tal que $\cdot(x,y) = x \cdot y$, isto é, associa-se a cada par $(x,y)$ de números reais o respectivo produto $x\cdot y$. A função $\cdot$ é a operação binária conhecida como \emph{produto sobre $\mathbb{R}$}.
\end{exemplo}

\begin{exemplo}[Composição de funções]
	O mapa $\circ:\mathcal{F}_\mathbb{R}\times\mathcal{F}_\mathbb{R}\longrightarrow\mathcal{F}_\mathbb{R}$, em que $\mathcal{F}_\mathbb{R}$ representa o \emph{conjunto das funções de $\mathbb{R}$ em $\mathbb{R}$}, é a operação definida pela \emph{composição de funções $\circ(f,g) = f\circ g$ sobre $\mathcal{F}_\mathbb{R}$}.
\end{exemplo}

\begin{observacao}[Notação de operação]
	Usaremos a notação \(*(a,b) = a*b\), para simplificar a escrita de
	propriedades. Também, quando não houver ambiguidade, suprimiremos o simbolo da operação, fazendo $a*b = ab$.
\end{observacao}

\begin{exemplo}[Adição sobre $\mathbb{R}$]
	A função $+: \mathbb{R}\times\mathbb{R}\longrightarrow\mathbb{R}$ definida pela soma $x+y$ é a operação de \emph{adição sobre $\mathbb{R}$}.
\end{exemplo}

\begin{definicao}
	Para $a,b,c \; \in S$, uma operação binária $*$ é dita
	
	\begin{itemize}
		\item \emph{Associativa}, se $(a*b)*c = a*(b*c)$;
		\item \emph{Comutativa}, se \(a*b = b*a\).
	\end{itemize}
\end{definicao}

\begin{exemplo}[Multiplicação matricial]
	Seja $\mathbb{R}^{n\times n}$ o \emph{conjunto das matrizes reais quadradas de ordem $n$}. A operação $\times: \mathbb{R}^{n\times n}\times\mathbb{R}^{n\times n}\longrightarrow\mathbb{R}^{n\times n}$ é o \emph{produto matricial $\times(M,N) = M\times N$ sobre $\mathbb{R}^{n\times n}$}. Sabe-se que essa operação é associativa, visto que $(XY)Z = X(YZ), \ \forall X,Y,Z \in \mathbb{R}^{n\times n}$. Porém, como
	$$
	\begin{pmatrix}
		1&2\\
		3&4
	\end{pmatrix}
	\times
	\begin{pmatrix}
		5&6\\
		7&8
	\end{pmatrix}
	=
	\begin{pmatrix}
		19&22\\
		43&50
	\end{pmatrix}
	\neq
	\begin{pmatrix}
		5&6\\
		7&8
	\end{pmatrix}
	\times
	\begin{pmatrix}
		1&2\\
		3&4
	\end{pmatrix}
	=
	\begin{pmatrix}
		23&34\\
		31&46
	\end{pmatrix},
	$$
segue que $\times$ não pode ser comutativa.
\end{exemplo}

\begin{exemplo}[Potência em $\mathbb{N}$]
	Seja $f(x,y) = x^y$ a operação de \emph{potenciação sobre $\mathbb{N}$}. $f$ não é nem associativa, $$\text{pois }2^{(3^4)} = 2^{81} \neq (2^3)^4 = 2^{12},$$ nem comutativa, $$\text{pois }2^3 = 8 \neq 3^2 = 9.$$
\end{exemplo}

\begin{exemplo}[Adição]
	As adições sobre $\mathbb{N},\mathbb{Z},\mathbb{Q},\mathbb{R}$ ou $\mathbb{C}$ são operações tanto associativas quanto comutativas. Deixa-se ao leitor mostrar que isso é verdade.
\end{exemplo}

\begin{proposicao}
	Seja uma operação associativa dada sobre o conjunto
	\(S\). Há uma única forma de definir, para todo inteiro \(n\), um
	produto de \(n\) elementos \(a_1,\dots,a_n \in S\) (diremos
	\([a_1\dotsb a_n]\)) com as seguintes propriedades:
	
	\begin{enumerate}
		\def\labelenumi{\arabic{enumi}.}
		\item
		o produto \([a_1]\) de um elemento é o próprio elemento;
		\item
		o produto \([a_1a_2]\) de dois elementos é dado pela operação binária;
		\item
		para todo inteiro \(1\leq i\leq n\),
		\([a_1\dotsb a_n] = [a_1\dotsb a_i][a_{i+1}\dotsb a_n]\).
	\end{enumerate}
\end{proposicao}

\begin{proof}
	A demonstração dessa proposição é feita por indução em \(n\).
\end{proof}

\begin{definicao}[Elemento neutro à esquerda e à direita]
	Dizemos que \(e\in S\) é um \emph{elemento neutro à direita para uma operação binária $*$} se \(e*a = a\) para todo \(a\in S\). Caso $a*e = a$ para todo $a\in S$, diremos que $e$ é um \emph{elemento neutro à direita para a operação binária $*$}.
\end{definicao}

\begin{exemplo}
	$0$ é o elemento neutro à direita e à esquerda da adição nos naturais, visto que para todo $a\in \mathbb{N}$ $a+0 = a$ e $0+a = a$.
\end{exemplo}

\begin{definicao}[Elemento neutro]
	Dizemos que \(e\in S\) é um \emph{elemento neutro para uma operação binária} (ou, também, uma \emph{identidade}) se \(ea = ae = a\) para todo \(a\in S\).
\end{definicao}

\begin{proposicao}
	O elemento identidade é único.
\end{proposicao}

\begin{proof}
	Se \(e,e'\) são identidades, já que \(e\) é identidade, então \(ee' = e'\) e, como $e'$ é uma identidade, \(ee' = e\). Logo \(e = e'\), isto é, a identidade é única.
\end{proof}

\begin{exemplo}
	Seja a matriz 
	$$ I_2 =
	\begin{pmatrix}
		1&0\\
		0&1
	\end{pmatrix}.
	$$ 
	Esta matriz é o elemento neutro da operação de multiplicação matricial sobre $\mathbb{R}^{2\times 2}$.
\end{exemplo}

\begin{observacao}
	Usaremos $\vec{1}$ para representar a identidade multiplicativa e $\vec{0}$ para denotar a aditiva.
\end{observacao}

\begin{definicao}[Elemento oposto]
	Seja uma operação binária que possua uma identidade. Um elemento \(a\in S\) é chamado \emph{invertível} se há um outro elemento \(b\in S\) tal que \(ab = ba = 1\). Desde que \(b\) exista, ela é única e a denotaremos por \(a^{-1}\) e a chamaremos
	\emph{inversa de $a$} (ou \emph{elemento oposto de $a$}).
\end{definicao}

\begin{exemplo}
	Em $\mathbb{R}$, 3 é um elemento invertível para a multiplicação, pois
	$$\frac{1}{3}\cdot3 = \vec{1} =  3\cdot\frac{1}{3}.$$
	Porém, $0$ não é invertível para essa operação, visto que não existe $n\in\mathbb{R}$ tal que $0\cdot n = \vec{1} = 1$.
\end{exemplo}

\begin{proposicao}\label{prop:compinv}
	Se \(a,b\in S\) possuem inversa, então a inversa da composição \((ab)^{-1} = b^{-1}a^{-1}\).
\end{proposicao}

\begin{exemplo}
	Sejam as funções $f,g\in \mathcal{F}_\mathbb{R}$ tal que $f(x) = 3x-1$ e $g(x) = \frac x 3 + 2$. Como ambas as funções são bijetoras, segue que possuem inversas para a composição de funções dadas por $f^{-1}(x) = \frac{x+1}3$ e $g^{-1}(x) = 3x - 6$. Da proposição~\ref{prop:compinv}, $$(f\circ g)^{-1} = g^{-1}\circ f^{-1} = g^{-1}(f^{-1}(x)) = g^{-1}(f^{-1}(x)) = 3\frac{x + 1}3 - 6 = x - 5.$$
\end{exemplo}

\begin{observacao}[Notação de potências]
	Usaremos as seguintes notações:
	\begin{itemize}
		\item \(a^n = a^{n-1}a\) é a operação de \(a\dotsb a\) \(n\) vezes;
		\item \(a^{-n}\) é a inversa de \(a^n\);
		\item \(a^0 = \vec{1}\).
	\end{itemize}
	Com isso, tem-se que \(a^{r+s} = a^ra^s\) e \((a^r)^s = a^{rs}\). (Isso
	não induz uma notação de fração \(\frac{b}{a}\) a menos que seja uma operação
	comutativa, visto que \(ba^{-1}\) pode ser diferente de \(a^{-1}b\)).
	Para falar de uma operação aditiva, usaremos \(-a\) no lugar de
	\(a^{-1}\) e \(na\) no lugar de \(a^n\).
\end{observacao}

\begin{definicao}[Elemento regular à direita e à esquerda]
	Dizemos que \(e\in S\) é um \emph{elemento regular à direita para uma operação binária $*$} se, para todo $x,y \in S$, \(e*x = e*y \implies x = y\). Caso \(x*e = y*e \implies x = y\) para todo $x,y\in S$, diremos que $e$ é um \emph{elemento regular à direita para a operação binária $*$}.
\end{definicao}

\begin{exemplo}
	$3$ é regular à direita e à esquerda da adição nos naturais, visto que para todo $x,y\in \mathbb{N}$, $3+x = 3 + y \implies x = y$ e $x + 3 = y+3 \implies x =y$.
\end{exemplo}

\begin{definicao}[Elemento regular]
	Dizemos que \(e\in S\) é um \emph{elemento regular para uma operação binária} se for regular tanto à esquerda quanto à direita.
\end{definicao}

\begin{exemplo}
	A matriz $\begin{pmatrix}
		1&2\\
		3&4
	\end{pmatrix}$ é 
\end{exemplo}


\phantomsection
\addcontentsline{toc}{chapter}{Referências Bibliográficas}

\bibliographystyle{unsrt}
\bibliography{../../references}

\end{document}