\documentclass[a4paper,12pt]{article}
\usepackage[a4paper,top=3cm,bottom=2cm,left=3cm,right=3cm,marginparwidth=1.75cm]{geometry}
\usepackage[brazil]{babel}
\usepackage[T1]{fontenc}
\usepackage[utf8]{inputenc}
\usepackage{amsmath}
\usepackage{amsthm}
\usepackage{MnSymbol}
\usepackage{wasysym}
\usepackage{hyperref}
\usepackage{color}
\definecolor{Blue}{rgb}{0,0,0.9}
\definecolor{Red}{rgb}{0.9,0,0}
\usepackage{esvect}
\usepackage{graphicx}
\usepackage{float}
\usepackage{indentfirst}
\usepackage{caption}
\usepackage{blkarray}
\newcommand\Mark[1]{\textsuperscript#1}
\usepackage{pgfplots}
\usepackage{amsfonts}
\usepackage[english, ruled, linesnumbered]{algorithm2e}
\usepackage{algorithmic}
\usepackage{enumitem}

\newcommand{\nucleoe}{\emph{\text{nu }}}
\newcommand{\nucleo}{\text{nu }}
\newcommand{\imageme}{\emph{\text{nu }}}
\newcommand{\imagem}{\text{nu }}

\theoremstyle{plain}
\newtheorem{teorema}{Teorema}[section]
\newtheorem{lema}{Lema}[section]
\newtheorem{proposicao}{Proposição}[section]
\newtheorem{corolario}{Corolário}[section]

\theoremstyle{definition}
\newtheorem{definicao}{Definição}[section]
\newtheorem{observacao}{Observação}[section]
\newtheorem{exemplo}{Exemplo}[section]

\newenvironment{solucao}
{\renewcommand\qedsymbol{$\triangle$}\begin{proof}[Solução]}{\end{proof}}

\title{\textsc{Operações Binárias e Aplicaćões}\\ \textsl{um resumo}}
\author{Guilherme Philippi}
\begin{document}
	\maketitle
	
	Apresenta-se nesse texto um compilado de definições e resultados envolvendo os conceitos de operações binárias e algumas de suas aplica. Tudo que aqui se apresenta fora extraído de \cite{johnAlgebra, michalAlgebra, maierAlgebra}, principalmente de \cite{johnAlgebra}.

\section{Operações binárias}

\begin{definicao}[Operação binária]
	Uma \emph{operação binária} sobre um conjunto \(S\) é uma função \(*: S\times S \longrightarrow S\).
\end{definicao}

\begin{observacao}[Notação de operação]
	Usaremos a notação \(*(a,b) = a*b\), para simplificar a escrita de
	propriedades. Também, quando não houver ambiguidade, suprimiremos o simbolo da operação, fazendo $a*b = ab$.
\end{observacao}


\begin{definicao}
	Para $a,b,c \; \in S$, uma operação binária $*$ é dita
	
	\begin{itemize}
		\item \emph{Associativa}, se $(a*b)*c = a*(b*c)$;
		\item \emph{Comutativa}, se \(a*b = b*a\).
	\end{itemize}
\end{definicao}

\begin{proposicao}
	Seja uma operação associativa dada sobre o conjunto
	\(S\). Há uma única forma de definir, para todo inteiro \(n\), um
	produto de \(n\) elementos \(a_1,\dots,a_n \in S\) (diremos
	\([a_1\dotsb a_n]\)) com as seguintes propriedades:
	
	\begin{enumerate}
		\def\labelenumi{\arabic{enumi}.}
		\item
		o produto \([a_1]\) de um elemento é o próprio elemento;
		\item
		o produto \([a_1a_2]\) de dois elementos é dado pela operação binária;
		\item
		para todo inteiro \(1\leq i\leq n\),
		\([a_1\dotsb a_n] = [a_1\dotsb a_i][a_{i+1}\dotsb a_n]\).
	\end{enumerate}
\end{proposicao}

\begin{proof}
	A demonstração dessa proposição é feita por indução em \(n\).
\end{proof}

\begin{definicao}
	Dizemos que \(e\in S\) é \emph{identidade} para uma operação binária se \(ea = ae = a\) para todo \(a\in S\).
\end{definicao}

\begin{proposicao}
	O elemento identidade é único.
\end{proposicao}
\begin{proof}
	Se \(e,e'\) são identidades, já que \(e\) é identidade, então \(ee' = e'\) e, como $e'$ é uma identidade, \(ee' = e\). Logo \(e = e'\), isto é, a identidade é única.
\end{proof}

\begin{observacao}
	Usaremos $\vec{1}$ para representar a identidade multiplicativa e $\vec{0}$ para denotar a aditiva.
\end{observacao}

\begin{definicao}[Elemento inverso]
	Seja uma operação binária que possua uma identidade. Um elemento \(a\in S\) é chamado \emph{invertível} se há um outro elemento \(b\in S\) tal que \(ab = ba = 1\). Desde que \(b\) exista, ela é única e a denotaremos por \(a^{-1}\) e a chamaremos
	\emph{inversa de $a$}.
\end{definicao}


\begin{proposicao}
	Se \(a,b\in S\) possuem inversa, então a composição \((ab)^{-1} = b^{-1}a^{-1}\).
\end{proposicao}

\begin{observacao}[Potências]
	Usaremos as seguintes notações:
	\begin{itemize}
		\item \(a^n = a^{n-1}a\) é a composição de \(a\dotsb a\) \(n\) vezes;
		\item \(a^{-n}\) é a inversa de \(a^n\);
		\item \(a^0 = \vec{1}\).
	\end{itemize}
	
	Com isso, tem-se que \(a^{r+s} = a^ra^s\) e \((a^r)^s = a^{rs}\). (Isso
	não induz uma notação de fração \(\frac{b}{a}\) a menos que seja uma operação
	comutativa, visto que \(ba^{-1}\) pode ser diferente de \(a^{-1}b\)).
	Para falar de uma operação aditiva, usaremos \(-a\) no lugar de
	\(a^{-1}\) e \(na\) no lugar de \(a^n\).
\end{observacao}


\phantomsection
\addcontentsline{toc}{chapter}{Referências Bibliográficas}

\bibliographystyle{unsrt}
\bibliography{../../references}

\end{document}