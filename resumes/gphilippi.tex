 \documentclass[a4paper,12pt]{article}
\usepackage[a4paper,top=3cm,bottom=2cm,left=3cm,right=3cm,marginparwidth=1.75cm]{geometry}
\usepackage[brazil]{babel}
\usepackage[T1]{fontenc}
\usepackage[utf8]{inputenc}
\usepackage{amsmath}
\usepackage{amsthm}
\usepackage{MnSymbol}
\usepackage{wasysym}
\usepackage{hyperref}
\usepackage{color}
\definecolor{Blue}{rgb}{0,0,0.9}
\definecolor{Red}{rgb}{0.9,0,0}
\usepackage{esvect}
\usepackage{graphicx}
\usepackage{float}
\usepackage{indentfirst}
\usepackage{caption}
\usepackage{blkarray}
\newcommand\Mark[1]{\textsuperscript#1}
\usepackage{pgfplots}
\usepackage{amsfonts}
\usepackage[english, ruled, linesnumbered]{algorithm2e}
\usepackage{algorithmic}
\usepackage{enumitem}

\newcommand{\nucleoe}{\emph{\text{nu }}}
\newcommand{\nucleo}{\text{nu }}
\newcommand{\imageme}{\emph{\text{nu }}}
\newcommand{\imagem}{\text{nu }}

\theoremstyle{plain}
\newtheorem{teorema}{Teorema}[section]
\newtheorem{lema}{Lema}[section]
\newtheorem{proposicao}{Proposição}[section]
\newtheorem{corolario}{Corolário}[section]

\theoremstyle{definition}
\newtheorem{definicao}{Definição}[section]
\newtheorem{observacao}{Observação}[section]
\newtheorem{exemplo}{Exemplo}[section]

\newenvironment{solucao}
{\renewcommand\qedsymbol{$\triangle$}\begin{proof}[Solução]}{\end{proof}}

\title{\textsc{Relações e Operações Binárias}\\ \textsl{um resumo}}
\author{Guilherme Philippi}
\begin{document}
\maketitle
	
Apresenta-se nesse texto um compilado de definições e resultados envolvendo os conceitos de relações entre conjuntos e operações binárias. Tudo que aqui se apresenta fora extraído de \cite{johnAlgebra, michalAlgebra, maierAlgebra}, principalmente de \cite{johnAlgebra}.
	
\section{Produto Cartesiano e Relações}

\begin{definicao}[Produto cartesiano]
	Sejam $A$ e $B$ conjuntos. O conjunto $$A\times B = \{(a,b) \ | \ a\in A \text{ e } b\in B\}$$
	é o \emph{produto cartesiano de A e B}.
\end{definicao}

\begin{exemplo}
	Se $A = \{1,2,3\}$ e $B = {3,4}$, então $$A\times B = \{(1,3),(1,4),(2,3),(2,4),(3,3),(3,4)\}.$$
\end{exemplo}

\begin{definicao}[Relação]
	Uma	\emph{relação} entre dois conjuntos $A$ e $B$ é um subconjunto $\mathcal{R}\subset A\times B$. Lê-se $(a,b) \in \mathcal{R}$ como ``$a$ está relacionado com $b$'' e escreve-se $a\mathcal{R}b$.
\end{definicao}

\begin{exemplo}[Relação de igualdade]\label{ex:igualdade}
	A realação $=$, chamada \emph{relação de igualdade}, é definida sobre um conjunto $S$ por $$= \text{é o subconjunto } \{(x,x) \ |\ x\in S\}\subset S\times S.$$
\end{exemplo}

\begin{observacao}
	Sempre que uma relação for definida entre um conjunto $S$ e ele mesmo, como no exemplo~\ref{ex:igualdade}, diremos que esta é uma relação \emph{sobre} $S$.
\end{observacao}

\begin{definicao}[Função]
		Uma \emph{função} $\varphi$ que mapeia $X$ em $Y$ é uma relação entre $X$ e $Y$ com a propriedade de que cada $x\in X$ só irá aparecer uma única vez, e exatamente uma, em um par ordenado $(x,y)\in \varphi$. Também chamamos $\varphi$ de \emph{mapa} ou \emph{mapeamento} de $X$ em $Y$. Escrevemos $\varphi: X\longrightarrow Y$ e expressaremos $(x,y)\in\varphi$ por $\varphi(x) = y$. O \emph{domínio} de $\varphi$ é o conjunto $X$ e o conjunto $Y$ é dito \emph{contradomínio} de $\varphi$. Chama-se de \emph{alcance} de $\varphi$ o conjunto $\varphi[X] = \{\varphi(x)\ | \ x \in X\}.$
\end{definicao}

\begin{definicao}[Função injetiva e sobrejetiva]
		Uma função $\varphi: X \longrightarrow Y$ é \emph{injetiva} se $\varphi(x_1) = \varphi(x_2) \iff x_1 = x_2$. Também, $\varphi$ é dita \emph{sobrejetiva} se o alcance de $\varphi$ é $Y$. Se uma função é injetiva e sobrejetiva, então dizemos que a função é \emph{bijetiva}.
\end{definicao}

\begin{definicao}
	Sejam $S$ um conjunto $\mathcal{R}$ uma relação sobre $S$.
	Dizemos que $\mathcal{R}$ é uma relação
	\begin{enumerate}
		\item \emph{(reflexiva).} se $a\mathcal{R}a$, para todo $a\in S$;
		\item \emph{(simétrica).} se para todo $a,b \in S$ $a\mathcal{R}b \iff b\mathcal{R}a$;
		\item \emph{(antissimétrica).} se $a\mathcal{R}b$ e $b\mathcal{R}a \implies a = b$, para todo $a,b \in S$;
		\item \emph{(transitiva).} se $a\mathcal{R}b$ e $b\mathcal{R}c \implies a\mathcal{R}c$, $\forall \ a,b,c\in S$.
	\end{enumerate}
\end{definicao}


\section{Relações de Equivalência e Partições}

\begin{definicao}[Partições]
	Seja \(S\) um conjunto. Uma \emph{particão} \(P\) de
	\(S\) é uma subdivisão de \(S\) em subconjuntos não vazios e não
	sobrepostos, isto é, uma união de conjuntos disjuntos.	
\end{definicao}

\begin{exemplo}
	Pode-se particionar o conjunto dos números inteiros
	\(\mathbb{Z}\) na união de disjuntos \(P\cup I\), onde
	\(P = \{z \in \mathbb{Z} \ |\ z \text{ é par}\}\) e
	\(I = \{z \in \mathbb{Z} \ |\ z \text{ é impar}\}\).
\end{exemplo}

\begin{definicao}[Relação de equivalência]
	Uma \emph{relação de equivalência} $\sim$ sobre um conjunto
	\(S\) é uma relação que precisa ser, para todo $a,b,c\in S$,
	\begin{enumerate}
		\item \emph{(Transitiva).} Se \(a\sim b\) e \(b\sim c\), então \(a\sim c\);
		\item \emph{(Simétrica).} Se \(a\sim b\), então \(b\sim a\);
		\item \emph{(Reflexiva).} \(a\sim a\).
	\end{enumerate}
\end{definicao}

\begin{observacao}
	A noção de partição em \(S\) e a relação de equivalência em \(S\) são
	lógicamente equivalentes: Dada uma partição \(P\) sobre \(S\), pode-se
	definir uma relação de equivalência \(R\) tal que, se \(a\) e \(b\)
	estão no mesmo subconjunto partição, então \(a\sim b\) e, dada uma
	relação de equivalência \(R\), podemos definir uma partição \(P\) tal
	que o subconjunto que contêm \(a\) é o conjunto de todos os elementos
	\(b\) onde \(a\sim b\). Esse subconjunto é chamado de \emph{classe de
		equivalência de \(a\)}
	\[C_a = \{b\in S \ | \ a\sim b\}\]
	e \(S\) é particionado em classes de equivalência.
\end{observacao}

\begin{proposicao}
	Sejam \(C_a\) e \(C_b\) duas classes de equivalência do conjunto \(S\). Se existe \(d\) tal que \(d\in C_a\) e \(d\in C_b\), então \(C_a = C_b\).
\end{proposicao}

\begin{observacao}[Representante]
	Seja um conjunto \(S\). Suponha que exista uma relação de equivalência
	ou uma partição sobre \(S\). Então, pode-se construir um novo conjunto
	\(\bar{S}\) formado pelas classes de equivalência ou os subconjuntos
	partições de \(S\). Essa construção induz uma notação muito útil: para
	\(a\in S\), a classe de equivalência de \(a\) ou o subconjunto partição
	que contém \(a\) serão denotados como o elemento
	\(\bar{a} \in \bar{S}\). Desta forma, a notação \(\bar{a} = \bar{b}\)
	significa que \(a \sim b\) e chamamos \(a,b \in S\) de
	\emph{representantes} das respectivas classes de equivalência
	\(\bar{a}, \bar{b} \in \bar{S}\).
\end{observacao}

\begin{definicao}
	Seja um mapeamento \(\varphi: S \longrightarrow T\).
	Chama-se de \emph{relação de equivalência determinada por \(\varphi\)} a
	relação dada por \(\varphi(a) = \varphi(b) \Rightarrow a \sim b\). Além
	disso, para um elemento \(t\in T\), o subconjunto de
	\(\varphi^{-1}(t) = \{s \in S\ | \ \varphi(s) = t\}\) é dito
	\emph{imagem inversa de \(t\) por \(\varphi\)}.
\end{definicao}

\begin{proposicao}
	Seja um mapeamento \(\varphi: S \longrightarrow T\)
	e \(t \in T\) um elemento qualquer de \(T\). Se a imagem inversa
	\(\varphi^{-1}(t)\) é não vazia, então \(t \in \text{im}\ \varphi\) e
	\(\varphi^{-1}(t)\) forma uma classe de equivalência
	\(\bar{\varphi}\in \bar{S}\) através da relação determinada por
	\(\varphi\).	
\end{proposicao}

\section{Operações binárias}

\begin{definicao}[Operação binária]
	Uma \emph{operação binária} sobre um conjunto \(S\) é uma função \(*: S\times S \longrightarrow S\).
\end{definicao}

\begin{observacao}[Notação de operação]
	Usaremos a notação \(*(a,b) = a*b\), para simplificar a escrita de
	propriedades. Também, quando não houver ambiguidade, suprimiremos o simbolo da operação, fazendo $a*b = ab$.
\end{observacao}


\begin{definicao}
	Para $a,b,c \; \in S$, uma operação binária $*$ é dita
	
	\begin{itemize}
		\item \emph{Associativa}, se $(a*b)*c = a*(b*c)$;
		\item \emph{Comutativa}, se \(a*b = b*a\).
	\end{itemize}
\end{definicao}

\begin{proposicao}
	Seja uma operação associativa dada sobre o conjunto
	\(S\). Há uma única forma de definir, para todo inteiro \(n\), um
	produto de \(n\) elementos \(a_1,\dots,a_n \in S\) (diremos
	\([a_1\dotsb a_n]\)) com as seguintes propriedades:
	
	\begin{enumerate}
		\def\labelenumi{\arabic{enumi}.}
		\item
		o produto \([a_1]\) de um elemento é o próprio elemento;
		\item
		o produto \([a_1a_2]\) de dois elementos é dado pela operação binária;
		\item
		para todo inteiro \(1\leq i\leq n\),
		\([a_1\dotsb a_n] = [a_1\dotsb a_i][a_{i+1}\dotsb a_n]\).
	\end{enumerate}
\end{proposicao}

\begin{proof}
	A demonstração dessa proposição é feita por indução em \(n\).
\end{proof}

\begin{definicao}
	Dizemos que \(e\in S\) é \emph{identidade} para uma operação binária se \(ea = ae = a\) para todo \(a\in S\).
\end{definicao}

\begin{proposicao}
	O elemento identidade é único.
\end{proposicao}
\begin{proof}
	Se \(e,e'\) são identidades, já que \(e\) é identidade, então \(ee' = e'\) e, como $e'$ é uma identidade, \(ee' = e\). Logo \(e = e'\), isto é, a identidade é única.
\end{proof}

\begin{observacao}
	Usaremos $\vec{1}$ para representar a identidade multiplicativa e $\vec{0}$ para denotar a aditiva.
\end{observacao}

\begin{definicao}[Elemento inverso]
	Seja uma operação binária que possua uma identidade. Um elemento \(a\in S\) é chamado \emph{invertível} se há um outro elemento \(b\in S\) tal que \(ab = ba = 1\). Desde que \(b\) exista, ela é única e a denotaremos por \(a^{-1}\) e a chamaremos
	\emph{inversa de $a$}.
\end{definicao}


\begin{proposicao}
	Se \(a,b\in S\) possuem inversa, então a composição \((ab)^{-1} = b^{-1}a^{-1}\).
\end{proposicao}

\begin{observacao}[Potências]
	Usaremos as seguintes notações:
	\begin{itemize}
		\item \(a^n = a^{n-1}a\) é a composição de \(a\dotsb a\) \(n\) vezes;
		\item \(a^{-n}\) é a inversa de \(a^n\);
		\item \(a^0 = \vec{1}\).
	\end{itemize}
	
	Com isso, tem-se que \(a^{r+s} = a^ra^s\) e \((a^r)^s = a^{rs}\). (Isso
	não induz uma notação de fração \(\frac{b}{a}\) a menos que seja uma operação
	comutativa, visto que \(ba^{-1}\) pode ser diferente de \(a^{-1}b\)).
	Para falar de uma operação aditiva, usaremos \(-a\) no lugar de
	\(a^{-1}\) e \(na\) no lugar de \(a^n\).
\end{observacao}

\phantomsection
\addcontentsline{toc}{chapter}{Referências Bibliográficas}

\bibliographystyle{unsrt}
\bibliography{references}

\end{document}